\chapter{Метрики интерпретируемости}

Из-за того, что для понятий "объяснения" и "интерпретации" не существует формального определения, встает очень большая проблема как оценивать качество объяснений, предложенных тем или иным методом, и как эти методы сравнивать между собой. В  рассмотрены различные методики оценивания и выделены три категории: оценивания приложением, людьми и функциональное оценивание (Application-grounded, Human-grounded, Functually-grounded соответственно).

\begin{itemize}
    \item \textbf{Application-grounded оценивание}. Суть в том, что если у нас есть реальная задача с ML-системой асистирующей человеку, можно оценивать то, насколько хорошо система взаимодействует с человеком. Хорошим примером является сегментация на снимках томографии. Объяснения работают хорошо, если они сокращают время работы эксперта, который работает с системой, позволяют находить ошибки и т. д.
    Очевидная проблема этого метода оценивания заключается в том, что он требует работы всей системы в production-среде. Более того, это требует трат времени и сил доменного эксперта, что делает этот вид оценивания наиболее дорогостоящим. 
    \item \textbf{Human-grounded оценивание}. Суть его в том, что человеку даются \textit{простые} задачи, по которым мы пытаемся понять ценность интерпретации модели. Главное отличие от Application-grounded оценивания заключается в том, что не требуется дорогостоящее время эксперта, чтобы протестировать заданную методику. Примеры экспериментов:
    \begin{itemize}
        \item Бинарный выбор, при котором человек должен из пары представленных объяснений выбрать более качественное
        \item Симуляция инференса. Человек при заданном объяснении и входном наблюдении должен корректно "просимулировать" модель и угадать её предсказание
        \item Симуляция контрфактов. Человек при заданном объяснении, входе и выходе модели должен ответить, как следует поменять вход так, чтобы изменилось предсказание модели.
    \end{itemize}
    \item \textbf{Functually-grounded оценивание}. Выбирается простая вычислимая прокси-метрика, по которой мы опосредованно судим об интерпретируемости модели. Так как её можно считать автоматически, без участия ручного труда человека, её очень удобно использовать для первичного оценивания методик. Её недостаток в том, что зачастую очень сложно выбрать работающую прокси-метрику. Для каждой выбранной прокси-метрики требуется веское обоснование, почему это можно использовать.
    Пример прокси-метрики - разреженность эмбеддингов, если ранее было показано, что разреженные модели лучше интерпретируются
\end{itemize}

% TODO: подвести итог